\documentclass{physics_notes}

\title{Quantum Mechanics 2}
\author{St Aidan's Physics Society}
\date{\today}

\usepackage{physics}

\begin{document}

\maketitle
\begin{figure*}[h!]
	\centering
	\includegraphics[width=\linewidth]{Figures/mich.png}
\end{figure*}
\tableofcontents
\newpage

\section*{Introduction}

These notes are designed to summarise the Foundations of Physics 2A (PHYS2581) Quantum Mechanics course, as taught by Prof. S Cole in the 2018/19 academic year. 

\section*{Summary}

\subsection*{The Five Postulates of Quantum Mechanics}

The content in this course can be summarised with the following five postulates:

\begin{enumerate}
	\item All possible predictions of the physical properties of a dynamical system can be obtained via its wavefunction.
	\item Every dynamical variable can be represented by a Hermitian operator, whose eigenvalues give the possible values of measurement of the dynamical variable. 
	\item The operators representing position and momentum are $x$ and $-i\hbar\nabla$ respectively. Operators for other dymanical variables can be derived using their corresponding dynaimcal variables' classical relations to position and momentum. 
	\item The probability of measuring a particular result of a superposition of states is given by the probability amplitude of the relevant eigenfunction, squared. 
	\item The time evolution of the wavefunction is described by the time-dependent Schr\"odinger equation.
\end{enumerate}

\section{Operators}\label{sec:operators}

All physical quantities in Quantum Mechanics can be represented by operators, which are essentially functions mapping from one space of physical states to another. From the $\hat{x} = x$ and $\hat{p} = -i\hbar\frac{\partial}{\partial x}$ operators along with time $t$, all other operators can be derived. For example, $\hat{E_k} = \frac{\hat{p}^2}{2m}$, as $E_k = \frac{p^2}{2m}$ classically. 

In quantum mechanics, these operators are made to act on wavefunctions: complex functions that describe everything we know about a system. The measureable values of the quantity represented by the general operator $\hat{Q}$ are the eigenvalues $q$ in the following eigenvalue problem:

\[ \hat{Q}\Psi = q\Psi \]

where $\Psi$ is the wavefunction of the system being measured. 

Classically, the Hamiltonian is usually the sum of kinetic and potential energies of a system. As such, we can define the Hamiltonian operator:

\begin{align*} 
\hat{H} &= \frac{\hat{p}^2}{2m} + V  \\
&= -\frac{\hbar^2}{2m} \nabla^2 + V
\end{align*}

Using $\hat{H}{\Psi} = E\Psi$ (that the eigenvalues of the Hamiltonian are the allowed energy levels of the system), we obtain the Schr\"odinger equation. 


\subsection{Expectation}

The expectation of a general operator $\hat{Q}$ in 3D is given by:

\[ \expval{\hat{Q}} = \int_{V} \Psi^*\hat{Q}\Psi dV \]

The volume element $dV$ varies depending on the coordinate system. The volume element in some common 3D coordinate systems are given below. 

\begin{table}[h!]
\centering
	\begin{tabular}{c|c|c}
	Coordinate system & Position vector & Volume element \\ \hline
	Cartesian & $x\hat{\imath} + y\hat{\jmath} + z\hat{k}$ & $dxdydz$ \\
	Spherical polar & $r\sin{\theta}\cos{\phi}\hat{\imath} + r\sin{\theta}\sin{\phi}\hat{\jmath} + r\cos{\theta}\hat{k}$ & $r^2\sin{\theta}drd\theta d\phi$ \\
	Cylindrical polar & $r\cos{\theta}\hat{\imath} + r\sin{\theta}\hat{\jmath} + z\hat{k}$ & $r dr d\theta dz$
	\end{tabular}
	\caption{Volume element in common coordinate systems}
\end{table}
\subsection{Hermitian Operators}

If an operator $Q$ is Hermitian, then $\bra{\psi}Q\ket{\psi} = \bra{\psi}Q\ket{\psi}^*$. All operators that represent a real, measurable quantity are Hermitian as if $\expval{\hat{Q}}$ is a measurable quantity, then we require that $\expval{\hat{Q}} = \expval{\hat{Q}}^*$, i.e. $\expval{\hat{Q}}$ is real, and we can show that this property is equivalent to the Hermitian property (proof given in Appendix A of course notes). As such:

\[ \bra{\psi}\hat{Q}\ket{\psi} =  \bra{\psi}Q\ket{\psi}^* \iff \expval{\hat{Q}} = \expval{\hat{Q}}^* \]

Hence all operators that represent a real, measurable quantity are Hermitian. 

\subsection{Commutators}

The commutator of two operators $\hat{A}$, $\hat{B}$ is defined as $[\hat{A},\hat{B}] = \hat{A}\hat{B} - \hat{B}\hat{A}$. It follows that $[\hat{A},\hat{B}] = -[\hat{B},\hat{A}]$. If $[\hat{A},\hat{B}] =[\hat{B},\hat{A}] = 0$, then $\hat{A}$ and $\hat{B}$ are said to commute, meaning they share a common set of eigenfunctions.  Some important commutator identities are given below:

\begin{itemize}
	\item $[\hat{A},\hat{A}] = 0$
	\item $[\hat{A} + \hat{B}, \hat{C}] = [\hat{A},\hat{C}] + [\hat{B},\hat{C}]$
	\item $[\hat{A}\hat{B},\hat{C}] = \hat{A}[\hat{B},\hat{C}] + [\hat{A},\hat{C}]\hat{B}$
	\item $[\hat{A},\hat{B}\hat{C}] = [\hat{A},\hat{B}]\hat{C} + \hat{B}[\hat{A},\hat{C}]$
\end{itemize}

\subsection{The Laplacian}

An operator of particular importance in Quantum Mechanics is the Laplacian, given below in 3D Cartesian, cylindrical polar and spherical polar coordinates respectively:

\begin{itemize}
\item 3D Cartesian \[\nabla^2 = \frac{\partial^2}{\partial x^2} + \frac{\partial^2}{\partial y^2} + \frac{\partial^2}{\partial z^2} \]
\item Cylindrical polar \[\nabla^2 = \frac{1}{r}\frac{\partial}{\partial r}\left(r\frac{\partial}{\partial r}\right) + \frac{1}{r^2}\frac{\partial^2}{\partial \theta^2} + \frac{\partial^2}{\partial z^2} \]
\item Spherical polar \[ \nabla^2 = \frac{1}{r^2}\frac{\partial}{\partial r}\left(r^2\frac{\partial}{\partial r}\right) + \frac{1}{r^2 \sin{\theta}} \frac{\partial}{\partial \theta} \left(\sin{\theta}\frac{\partial}{\partial\theta}\right) + \frac{1}{r^2 \sin{\theta}^2} \frac{\partial^2}{\partial \phi^2} \]
\end{itemize}



\section{Wave functions}\label{sec:wave_functions}
\subsection{Probability distributions}
\subsection{Superposition}
\subsection{Dirac Notation}

In Dirac notation, devised by Paul Dirac, bra and ket vectors are used to represent wave functions and their complex conjugates:

\[ \text{(ket):\;} \ket{\Psi} = \Psi; \; \text{(bra):\;} \bra{\Psi} = \Psi^*; \]

They can be used together to represent the expectation of an operator $\hat{Q}$:

\[ \expval{\hat{Q}}{\Psi} = \int_{V} \Psi^*\hat{Q}\Psi dV \]
\subsection{Eigenfunctions of various potentials}

\section{Relativistic corrections}\label{sec:relativistic_corrections}
\subsection{Non-degenerate perturbation theory}
\subsection{Degenerate perturbation theory}
\subsection{The Hydrogen atom}
\subsubsection{Degenerate perturbation theory in hydrogen}
\subsubsection{Hydrogen fine splitting}




\end{document}
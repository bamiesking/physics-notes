\subsection{Eigenfunctions of various potentials}

\subsubsection{The infinite square well}

The infinite square well is an ideal quantum potential given in one dimension by

\[ V(x) = \left\{\begin{matrix}
\infty & \text{if} & x < 0 \\
0 & \text{if} & 0\leq x \leq L \\
\infty & \text{if} & x > L 
\end{matrix}\right. \]

Hence the Schr\"odinger equation in an infinite potential well is:

\[ -\frac{\hbar^2}{2m}\frac{d^2 \psi}{dx^2} = E\psi \]

Solving this 2nd order ODE and applying the boundary condition $\psi(0) = 0$ gives:

\[ \psi(x) = A\sin(kx)\text{, where } k = \frac{\sqrt{2mE}}{\hbar} \]

Normalising and applying the boundary condition $\psi(L) = 0$ gives

\[ \psi_n (x) = \sqrt{\frac{2}{L}}\sin(\frac{n\pi x}{L}) \text{ with } E = \frac{n^2 \pi^2 \hbar^2}{2m L^2} \]

Adding the time dependence leads to the full energy eigenfunctions:

\[ \Psi_n (x,t) = \psi_n (x) e^{-\frac{iE_n t}{\hbar}} = \sqrt{\frac{2}{L}}\sin(\frac{n\pi x}{L}) e^{-\frac{iE_n t}{\hbar}} \]




\subsubsection{The linear harmonic oscillator}

\subsection{Finding the Hydrogen Wavefunction}

\subsubsection*{Hydrogen atom in a magnetic field}
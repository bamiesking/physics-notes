\subsection{The quantum wavefunction}

A consequence of waave-particle duality is that what would classically be considered as discrete particles, such as electrons, can be described via a wavefunction (analagous to a classical wave equation). In order for an arbitrary wavefunction $\Psi(\vec{r}, t)$ to be valid, it has to obey the Schr\"odinger equation (given here in its 3D time-dependent form):

\[ \hat{H}\Psi = i\hbar\frac{\partial\Psi}{\partial t} \]

where $\hat{H} = -\frac{\hbar^2}{2m}\nabla^2 + V(\vec{r}, t)$ is the Hamiltonian operator defined previously. A useful alternate form of the equation is the 1D time-independent Schr\"odinger equation:

\[ \hat{H}\Psi = E\Psi \]

where $E$ is the total energy of the system, and the Hamiltonian in 1D is $\hat{H} = -\frac{\hbar^2}{2m}\frac{\partial^2}{\partial x^2} + V(x)$.

\subsubsection*{Probability distributions and Normalisation}

A further requirement for a wavefuntion to be valid in real space is for the total probability of finding the particle at some position in space is equal to 1. The probability of finding a particle described by the complex unnormalised wavefunction $f(x,t)$ between $x$ and $x+dx$ is:

\[ P(x,t)dx \propto \left|f(x,t)\right|^2 dx = f^*(x,t)f(x,t) dx \]

where the constant of proportionality is the normalisation constant $N$, given by:

\[ N = \frac{1}{\sqrt{\int_{-\infty}^{+\infty} \left|f(x,t)\right|^2 dx}} \]

So $\Psi(x,t) = Nf(x,t)$ and $\int_{-\infty}^{+\infty} P(x,t)dx = \int_{-\infty}^{+\infty} \Psi^*(x,t)\Psi(x,t)dx = 1$.

\subsubsection*{Dirac notation}

In Dirac notation, devised by Paul Dirac, bra and ket vectors are used to represent wave functions and their complex conjugates:

\begin{align*} 
\text{(ket):\;} &\ket{\psi} = \psi; \\ 
\text{(bra):\;} &\bra{\psi} = \psi^*; 
\end{align*}

They can be used together to represent the expectation of an operator $\hat{Q}$:

\[ \expval{\hat{Q}} = \expval{\hat{Q}}{\psi} = \int_{V} \psi^*\hat{Q}\psi dV \]

This alternative notation provides a very elegant way of dealing with the maths of Quantum Mechanics. 

\subsection{Eigenfunctions of the Schr\"odinger equation}

We have seen the time-independent Schr\"odinger equation expressed in the form of an eigenvalue problem:

\[ \hat{H}\psi_E(x) = E\psi_E(x) \]

Where $E$ is the eigenvalue corresponding the eigenfunction (often referred to as an eigenstate) $\psi_E(x)$. What we have previously referred to as the wavefunction $\Psi$ is actually the weighted sum of the eigenstates of the system with the time-dependece included, over all of the possible energy levels:

\[ \Psi(x,t) = \sum_E c_E \psi_E(x,t) e^{-\frac{iEt}{\hbar}} \]

This time-dependence of energy can be derived from the fact that the time-dependent Schr\"odinger equation is separable into two ODEs, giving the solution $\Psi_E(x,t) = X_E(x)T_E(t) = \psi_E(x)e^{-\frac{iEt}{\hbar}}$. Note that the total wavefunction given above is \emph{not} a solution of the time-independent Schr\"odinger equation. 

A useful property is that eigenfunctions corresponding to different eigenvalues are orthogonal, meaning that for normalised eigenfunctions $\psi_n$ and $\psi_m$ given by

\begin{align*}
\hat{H}\psi_n &= E_n\psi_n \\
\hat{H}\psi_m &= E_m\psi_m
\end{align*}

we have that $\int \psi_n \psi_m^* = \delta_{nm}$ where $\delta$ is the Kronecker delta. 

The eigenfunctions of an operator form a complete basis of the space containing every possible state of the system - this is explored in greater depth in the PHYS2631 Quantum Theory course. A system can exist in a superposition of several eigenstates, where each state has a probability of being measured. The act of measurement is said to `collapse' the wavefunction to the eigenstate corresponding to the measured value of the physical quantity in question, by making the probability of being found in all other eigenstates zero. 

We see that if $\psi(x) = \sum_E c_E \psi_E(x) $, then 

\[ \expval{E} = \sum_m |c_m|^2E_m \]

It turns out that $|c_m|^2$ is the probability of measuring the superposition of states to be in the state $\psi_m$ with energy $E_m$. We can calculate these $c_m$ using the following relation (a justification for this is given in the course notes, §6.2):

\[ c_m = \int\psi^*_m\psi dx = \bra{\psi_m}\ket{\psi}\]

Including the time-dependence of each state leads us to find that superpositions of eigenfunctions are not stationary states (that is, they vary with time) unlike eigenstates, which have no time-dependence. 


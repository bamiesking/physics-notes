\subsection{Systems: Degrees of Freedom, Constraints and Coordinates}
Any set of variables that can describe a \textit{configuration} of a system are dynamical variables. For cartesian coordinates, the dynamical variables are (x, y, z) and you may be familiar with spherical polar, $(r, \theta, \phi)$, and cylindrical polar, $(r, \theta, z)$. \par
Consider a particle at (2, -1, 0) in cartesian. If the particle had a velocity, $v$, then the configuration of the system will change, say, to (2, -1, 3) after 1 s. It could change under the action of a force, \textbf{F}, or have an inital speed. In general we have that the equation of motion of a system specifies the dynamical variables as time passes. 
%%inclusion of phase space needed? 
You may have come across \textit{phase space} before which shows all possible states of the system. There is an important distinction here: Configuration space is the vector space defined by the coordinates used whereas phase space is the possible configurations of the system. \par

Let's consider the particle described before. Assume that the particle has $\textbf{v} = 3\hat{k}$ m$s^{-1}$ . The configuration space is simply the 3D cartesian space. More formally, $\rm I\!R^{3}$. What would the phase space look like for the particle? We know the particle will only change its components of $\hat{k}$ as it is only travelling in that direction. Let's say that at $t=0$, the particle was at (2, -1, 0). Then it follows that it is a line in $\rm I\!R^{3}$ with $x=2$, $y=-1$ and $z>0$ with the direction of the line parallel to the $\hat{k}$ axis.\par

Why is configuration space useful? We can introduce the idea of constraints of a mechanical system to configuration space. An example of a constraint might be a ball confined to the surface of a table or a sphere. The configuration space for the latter would only be the surface of the sphere. Formally, it is the subset of the original $\rm I\!R^{3}$ particle to the sphere $S^{2}$. If we consider cartesian coordinates, for the former we have $x=y=constant$ and the latter $x^{2} + y^{2} + z^{2} = d^{2}$ where d is the radius of the sphere. These types of constraints are called \textit{Scleronomic} constraints. They are expressed in the form
\begin{equation}
f(x , y, z, w,....) = 0 
\end{equation}
Notice they only depend on the coordinates of the system and have no time dependence. This distinguishes it from the other type of constraint, \textit{Rhenomic} constraints. These have a time dependence in their expression.
\begin{equation}
f(x, y, z, w,....., t) = 0   
\end{equation}
An example of this would be a ball spinning, or a bead rotating on a wire. There is another form that isn't used much in the course. \textit{Nonholonomic} constraints involve differential equations or inequalities in their expression. A velocity dependent force is one such example. \par
What do constraints allow us to do? They're an algebraic expression so it comes naturally that they would be used to eliminate coordinates in the system. This leads on to the question of: \textit{How few coordinates do we need to entirely describe a system?}\par
For each independent dynamical variable, the system has that many \textit{degrees of freedom} (DoF). Starting with the simplest example, a point mass, $\textbf{r}(t)$, has 3 Dof. It can move freely in the x direction, y direction or z direction(or combination thereof). More generally, a system of $n$ point masses has $N = 3n$ DoF. We now have some independent constaints, $j$, to elimate some coordinates or DoF. This leads to:
\begin{equation}
N = 3n - j 
\end{equation}
\par To conclude, we use $q_{i}$ to denote a generalised coordinate throughout the notes. As a penultimate general expression, we can write the position of some particle in the system, $r_{i}$, as a function the generalised coordinates of the system.
\begin{equation}
r_{i} = r_{i}(q_{1}, q_{2}, ..., q_{N}, t)
\end{equation}
This is just to formalise everything in a neat expression. Since we have a generalised postion, we can find a generalised velocity, $\dot{r_{i}}$. This leads to an inequality after regarding $q_{i}$ and $\dot{q_{i}}$ as independent. This is so $\frac{\dot{q_{i}}}{q_{i}} =  0$
\begin{equation}
 \frac{\dot{r_{i}}}{\dot{q_{i}}} =  \frac{r_{i}}{q_{i}} 
\end{equation}
This is a nice expression to check whether the coordinates of the system are `right'.
\par


 

\title{\large{Exercises}}
\begin{enumerate}
	\item How many DoF are there for each of these systems?
	\begin{enumerate}
		\item 2 point masses connected by a rigid rod? 
		\item A circle rolling down a ramp?
		\item A pendulum swinging in one plane?
		\item The Earth and the Sun? 
	\end{enumerate}
	\item Derive (5) using (4) (Hint: Take the derivative with respect to $\dot{q_{i}}$)
	%% maybe add one or two more here

\end{enumerate}


\subsection{D'Alembert's Principle and The Langrangian}
You will be learning a new forumlation of mechanics as oppose to Newton's by resolving forces. To do this we need to assume a few things. (The point at which they're used will be stated.)
\begin{enumerate}
\item \textbf{Holonomic constraints!} We don't want any constraints with differentials in them; that means no friction or viscous drag...
\item \textbf{Constraining forces do no work.} In other words they're perpendicular to the motion. E.g. when a particle is moving across a table, the force restraining it is the reaction force (which is perpendicular to its direction of motion).
\item \textbf{Forces are conservative.} This will be well defined in Mathematical methods, but it's saying the potential of the force exsists. The total work done is independent of its path.
\end{enumerate}
\par 
In the beginning there was Newton's 2nd law, $\dot{p} = F_{i}$. For each force $F_{i}$ on the particle $i$, we take a \textit{virtual displacement} $\delta r_{i}$. 
\begin{equation}
\sum_{i} (F_{i} - \dot{p}) \cdot \delta r_{i} = 0
\end{equation}
The virtual displacement is some abitrary change of coordinates whilst time is held constant. It must respect the constraints of the system! Say we had a ball placed on a table in the $xy$ plane, we cannot have $z + \delta z$ because the ball is constrained to the table. The total force $F_{i}$ is the sum of the constraining force and the applied forces. Using our second assumption, we can rewrite (6) just in terms of the applied forces. This is known as D'Almbert's Principle.
\begin{equation}
\sum_{i} (F_{i}^{a} - \dot{p}) \cdot \delta r_{i} = 0
\end{equation}
We want to replace $r_{i}$ with a generalised form of coordinates. We know that each force, $F_{i}$ is some function of the independent coordinates. (Specifically related to its potential by $-\nabla V(r_{i}, ... r_{n})$)
After a \textit{long} derivation we arrive at the first milestone of Lagrangian mechanics. 
\begin{equation}
L = T - V
\end{equation}
Where T is the kinetic energy and V is the potential. L being the expression denoted the "Lagrangian". Be careful with signs of potential! Further note that we define the potential to be abitrary so that our expressions can differ by a constant term e.g. $3mg$. We will see briefly why this does not matter. But why is this useful? Using the $\textit{Euler-Lagrange Equation}$ we can find the equation of motion! 
\begin{equation}
\frac{d}{dt} (\frac{\partial L}{\partial \dot{q_{k}}}) - \frac{\partial L}{\partial q_{k}} = 0
\end{equation}
Let that sink in! This expression will yield the acceleration of $q_{k}$ only by considering the energies of the system! Rather then consider resultant force on a particle we can compactly find everything from this formula. \par 
To round off, if we have a time derivative of a coordinate in the Lagrangian, L, but the coordinate does not appear itself, this is called an ignorable coordinate. But why is it ignorable? If we consider (8) we find that 
\begin{equation}
\frac{d}{dt} (\frac{\partial L}{\partial \dot{q_{k}}}) = 0 \implies \frac{\partial L}{\partial \dot{q_{k}}} = constant
\end{equation}
What is the constant (10) is refering to? A constant of the motion, like linear momentum. More generally, the \textit{canonically conjugate momentum} to the coordinate $q_{k}$. Having a constant of motion can help simplify problems.
\par \vfill
\title{\large{Exercises}}
\begin{enumerate}
	\item How does having an abitrary potential not affect the equation of motion? 
	\item Why do we run into a problem with magnetic fields (with charged particles)? 
\end{enumerate}

\subsection{Euler's Formula and Variational Calculus}
We introduced Euler's formula as the magical solve all for systems. In order to find out where it comes from we need to consider the even more magical and general form of this equation and understand where it comes from, The Calculus of Variations. We'll begin with a \textit{very} general statement as all things in mathematics do and break everything down.
\title{What path y(x) leads to an extreme value of the functional}
\begin{equation}
I[y] = \int_{x_{1}}^{x_{2}} \! f(x, y, \frac{dy}{dx}) \textrm{d}x
\end{equation}

\begin{figure}[h!]
\title{where the trajectory goes through the point $(x_{1}, y_{1})$ and $(x_{2}, y_{2})$?}
\centering
\begin{tikzpicture}
\draw (0.8,1) node {$(x_{1}, y_{1})$};
\draw (2.1,4) node {$(x_{2}, y_{2})$};
\draw[domain=1.05:1.9,smooth,variable=\x] plot ({\x}, {\x * \x  });
\end{tikzpicture}
\caption{What path minimises the distance between the two points?}
\end{figure}
\begin{enumerate}
\item You should know what is meant by \textbf{path y(x)} and an extreme value. However, when we're looking at this method the word \textit{value} takes the physical meaning of whatever it is you are minimising (or maximising....). For example, distance. The word extreme is the mathematical sense of the word.
\item The functional is the expression inside the integral of (11). Generally, it means a function of a function(s). 
\item We want to `solve' this problem set to some initial conditions which is why we have $x_{1}, x_{2}$ and so on. The word trajectory here is the path of the function through the xy plane. Note that we're not minimising the trajectory!
\end{enumerate}
To summarise, we're changing the path, y(x), \textit{in order to} minimise $I[y]$, our elusive function. How do we do this?
\par
\title{\large{\textbf{Derivation}}}
As in most of calculus we'll consider a small change in the variable y so that our path changes and see what happens. \medskip
\begin{equation}
\begin{gathered}
\delta I = I[y +\delta y] - I[y] = \\ 
\int \! F(y + \delta y, \frac{dy}{dx} + \delta \frac{dy}{dx}, x) \textrm{d}x -  f(x, y, \frac{dy}{dx}) \textrm{d}x
\end{gathered}
\end{equation}
....
The point is that this leads to our holy grail equation.





\documentclass{physics_notes}

\usepackage{makeidx}
\usepackage{showidx}

\title{Integral Transforms}
%\author{St Aidan's Physics Society}
\date{\today}
\usetikzlibrary{decorations}
\usetikzlibrary{decorations.pathmorphing}
\usetikzlibrary{decorations.markings}
\usetikzlibrary{calc}

\newcommand{\intfty}{\int_{-\infty}^\infty}
\newcommand{\zintfty}{\int_0^\infty}
\newcommand{\F}[2]{\mathcal{F}\left[#1\right](#2)}
\renewcommand{\Finv}[2]{\mathcal{F}^{-1}\left[#1\right](#2)}
\newcommand{\intF}[3]{\frac{1}{\sqrt{2\pi}}\intfty #1 e^{i#3#2}d#2}
\newcommand{\intFinv}[3]{\frac{1}{\sqrt{2\pi}}\intfty #1 e^{-i#3#2}d#3}
\renewcommand{\L}[2]{\mathcal{L}\left[#1\right](#2)}
\newcommand{\Linv}[2]{\mathcal{L}^{-1}\left[#1\right](#2)}
\newcommand{\intL}[3]{\zintfty #1 e^{-#3#2}d#2}

\DeclareMathOperator{\sgn}{sgn}

\makeindex

\begin{document}

\maketitle
\tableofcontents
\newpage

\section*{Introduction}

These notes are based on the Integral Transforms course delivered in Epiphany 2020, by Dr Daniel Cerdeño. 

There are also frequent references, both explicit and otherwise, made to the book \emph{Mathematical Methods for Physics and Engineering} by K. F. Riley, M. P. Hobson, and S. J. Bence. Relevant chapters of the book are highlighted using coloured square brackets, e.g. \book{24.5}.

Contributors: Ben Amies-King

For a function $f: \mathbb{R} \rightarrow \mathbb{C}$, we define an integral transform:\index{integral transform}

\begin{equation}
g(\omega) = \int_a^b k(\omega, x) f(x) dx
\end{equation}

where the kernel (or propagator) $k(\omega, x)$ is a complex function of two real variables. \index{kernel}

\section{Fourier Transforms}\index{Fourier transform}

We define the Fourier transform, where:

\begin{equation*}
k(\omega, x) = \frac{1}{\sqrt{2\pi}} e^{i\omega x}
\end{equation*}

Hence:

\begin{equation}
\hat{f}(\omega) = \mathcal{F}[f(x)](\omega) = \frac{1}{\sqrt{2\pi}}\int_{-\infty}^\infty f(x) e^{i\omega x} dx
\end{equation}

The condition that $f(x)$ is sufficient to ensure that $\hat{f}(\omega)$ exists. An inverse transform $\mathcal{F}^{-1}: \hat{f} \rightarrow f$ exists if and only if $f$ is continuous. 

Note that the choice of normalisation factor and exponential sign is arbitrary. 

\subsection{Properties of the Fourier transform}

\begin{enumerate}
	\item{
	\begin{align*}
	\mathcal{F}\left[\frac{df}{dx}\right](\omega) &= \frac{1}{\sqrt{2\pi}} \int_{-\infty}^\infty \frac{df}{fx} e^{i\omega x} dx \\
	&= \left.\frac{f e^{i\omega x}}{\sqrt{2\pi}}\right|^{\infty}_{-\infty} - \frac{i\omega}{\sqrt{2\pi}}\int^\infty_{-\infty} f e^{i\omega x} dx \\
	%&= \footnote{0}{As $f$ is square-integrable} - i\omega \mathcal{F}\left[f(x)\right](\omega)
	&= 0 - i\omega \mathcal{F}\left[f(x)\right](\omega)
	\end{align*}
	}
	\item{
	\[\mathcal{F}\left[\frac{d^n f}{dx^n}\right](\omega) = (- i\omega)^n \mathcal{F}\left[f(x)\right](\omega)\]
	}
	\item{
	\begin{align*}
	\mathcal{F}\left[x f(x)\right](\omega) &= \frac{1}{\sqrt{2\pi}} \int_{-\infty}^\infty x f(x) e^{i\omega x} dx \\
	%&\footnote{=}{$\frac{\partial}{\partial\omega}\left(e^{i\omega x}\right) = ixe^{i\omega x}$} \frac{1}{\sqrt{2\pi}} \int_{-\infty}^\infty \frac{f(x)}{i} \frac{\partial}{\partial\omega}\left(e^{i\omega x}\right) dx\\
	&= \frac{1}{\sqrt{2\pi}} \int_{-\infty}^\infty \frac{f(x)}{i} \frac{\partial}{\partial\omega}\left(e^{i\omega x}\right) dx\\
	&= \frac{1}{i}\frac{d}{d\omega} \frac{1}{\sqrt{2\pi}}\int_{-\infty}^{\infty} f(x) e^{i\omega x} dx \\
	&= \frac{1}{i} \frac{d}{d\omega}\mathcal{F}\left[f(x)\right](\omega)
	\end{align*}
	}
	\item{
	\begin{align*}
	\mathcal{F}\left[f(ax)\right](\omega) &= \frac{1}{\sqrt{2\pi}} \int_{-\infty}^\infty f(ax)e^{i\omega x} dx \\
	&= \frac{1}{\sqrt{2\pi}} \int_{-\infty}^\infty f(x') e^{i\frac{\omega}{a}x'} \frac{dx'}{a} \\
	&= \frac{1}{a}\mathcal{F}\left[f(x)\right]\left(\frac{\omega}{a}\right)
	\end{align*}}
	\item{
	\begin{align*}
	\mathcal{F}\left[f(x+a)\right](\omega) &= \frac{1}{\sqrt{2\pi}} \int_{-\infty}^\infty f(x+a)e^{i\omega x} dx \\
	&= e^{-i\omega a}\frac{1}{\sqrt{2\pi}} \int_{-\infty}^\infty f(x') e^{i\omega (x'-a)}dx' \\
	&= e^{-i\omega a}\mathcal{F}\left[f(x)\right]\left(\omega\right)
	\end{align*}}
	\item{
	\begin{align*}
	\mathcal{F}\left[f(x) e^{iax}\right](\omega) &= \frac{1}{\sqrt{2\pi}} \int_{-\infty}^\infty f(x)e^{iax}e^{i\omega x} dx \\
	&= \frac{1}{\sqrt{2\pi}} \int_{-\infty}^\infty f(x) e^{i(a + \omega)x}dx \\
	&=\mathcal{F}\left[f(x)\right]\left(\omega + a\right)
	\end{align*}}
	\item{For $f^\star(x) = f(x)$:
	\begin{align*}
	\left(\mathcal{F}\left[f(x) e^{iax}\right](\omega)\right)^\star &= \left(\frac{1}{\sqrt{2\pi}} \int_{-\infty}^\infty f(x)e^{i\omega x} dx \right)^\star\\
	&= \frac{1}{\sqrt{2\pi}} \int_{-\infty}^\infty f^\star(x)e^{-i\omega x} dx \\
	&= \frac{1}{\sqrt{2\pi}} \int_{-\infty}^\infty f(x)e^{-i\omega x} dx \\
	&=\mathcal{F}\left[f(x)\right](-\omega)
	\end{align*}}
\end{enumerate}

\begin{example}{Wave equation}

\begin{equation*}
\frac{d^2y(x,t)}{dx^2} = \frac{1}{v^2}\frac{d^2y(x,t)}{dt^2}
\end{equation*}

We define \[\mathcal{F}[y(x,t)](k,t) = \hat{y}(k,t)\] so \[\mathcal{F}\left[\frac{d^2y(x,t)}{dx^2}\right](k,t) = (-ik)^2\hat{y}(k,t)\] Hence \[\frac{d^2\hat{y}(k,t)}{dt^2} = -(kv)^2\hat{y}(k,t)\] This yields the solution \[\hat{y}(k,t) = A(k)e^{ikvt} + B(k)e^{-ikvt}\]

We must then transform back into spacial coordinates:

\begin{align*}
y(x,t) &= \mathcal{F}^{-1}\left[A(k)e^{ikvt} + B(k)e^{-ikvt}\right](x,t) \\
&= \frac{1}{\sqrt{2\pi}}\int_{-\infty}^\infty \left(A(k)E^{ikvt} + B(k)e^{-ikvt}\right)e^{-ikx} dk \\
&= \frac{1}{\sqrt{2\pi}}\int_{-\infty}^\infty \left(A(k)E^{-ik(x - vt)} + B(k)e^{-ik(x + vt)}\right)dk \\
\end{align*}
\end{example}


\subsection{$n$-dimensional Fourier transform}

We can generalise this definition to an $n$-dimensional vector space:
\index{n dimensional Fourier transform}
\begin{equation}
\hat{f}(\vec{\omega}) = \frac{1}{(2\pi)^{\frac{n}{2}}} \int_{-\infty}^\infty f(\vec{x}) e^{i\vec{\omega}\cdot\vec{x}} d^n x
\end{equation}


\subsection{Connection to Fourier series}
\index{Fourier series}

Consider a periodic function $f(x)$ over $[-L,L]$. We can write:

\begin{equation}
f(x) = a_0 + \sum_{n=1}^\infty \left(a_n \cos{\frac{n\pi x}{L}} + b_n \sin{\frac{n\pi x}{L}}\right)
\end{equation}

with 

\begin{align*}
a_0 &= \frac{1}{2L}\int_{-L}^{L} f(t) dt \\
a_n &= \frac{1}{L}\int_{-L}^{L} f(t) \cos{\frac{n\pi t}{L}} dt \\
b_n &= \frac{1}{L}\int_{-L}^{L} f(t) \sin{\frac{n\pi t}{L}} dt
\end{align*}

Substituting these expressions into the definiton of the series and applying the identity:

\begin{equation*}
\cos{\frac{n\pi x}{L}}\cos{\frac{n\pi t}{L}} + \sin{\frac{n\pi x}{L}}\sin{\frac{n\pi t}{L}} = \cos{\frac{n\pi}{L}(t-x)}
\end{equation*}

gives

\begin{align*}
f(x) &= \frac{1}{2L}\int_{-L}^{L} f(t)dt + \frac{1}{L}\sum_{n=1}^\infty \int_{-L}^L f(t) \cos{\frac{n\pi}{L}(t-x)} dt \\
&= \frac{1}{2L}\int_{-L}^{L} f(t)dt + \frac{1}{2L}\sum_{n=1}^\infty \int_{-L}^L f(t) \left[ e^{\frac{in\pi}{L}(t-x)} + e^{-\frac{in\pi}{L}(t-x)} \right] dt \\
&= \frac{1}{2L}\int_{-L}^L f(t) e^{\frac{in\pi}{L}(t-x)} \\
\intertext{Defining $\omega_n = \frac{n\pi}{L}$, $\delta\omega = \omega_{n-1} - \omega_n = \frac{\pi}{L}$}
&=\sum_{n=-\infty}^{\infty} \frac{\delta\omega}{2\pi} \int_{-L}^L f(t) e^{i\omega_n t} e^{-i\omega_n x} \\
&\to \frac{1}{\sqrt{2\pi}}\int_{\infty}^\infty \left(\frac{1}{\sqrt{2\pi}} \int_{-\infty}^\infty f(t) e^{i\omega t}dt\right) e^{-i\omega x} d\omega \\
\intertext{as $L\to\infty$}
\end{align*}

\begin{example}{Gaussian function}
Let $f(x) = e^{-\frac{x^2}{2}}$. Then

\begin{align*}
\mathcal{F}\left[f(x)\right](\omega) &= \frac{1}{\sqrt{2\pi}}\int_{-\infty}^\infty e^{-\frac{x^2}{2}} e^{i\omega x} dx \\
&= \frac{1}{\sqrt{2\pi}}\int_{-\infty}^\infty e^{\frac{1}{2}(x - i\omega)^2 - \frac{\omega^2}{2}} dx \\
&= \frac{e^{-\frac{\omega^2}{2}}}{\sqrt{2\pi}}\int_{-\infty}^\infty e^{-\frac{1}{2}(x - i\omega)^2} dx
\end{align*}

For a contour $C_R = C_1 + C_2 + C_3 + C_4$, as $e^{-\frac{z^2}{2}}$ is entire, we have $\oint_{C_R} e^{-\frac{z^2}{2}} = 0$ from Cauchy's theorem.

We need to evaluate 
\[ \lim_{R\to\infty} \left(\int_{C_1} + \int_{C_2} + \int_{C_3} + \int_{C_4}\right)e^{-\frac{z^2}{2}}dz\]

We see 
\begin{align*}
\int_{C_1} e^{-\frac{z^2}{2}} dz &\to \int_{-\infty}^\infty e^{-\frac{x^2}{2}} dx \\
\int_{C_2} e^{-\frac{z^2}{2}} dz &= \int_{C_4} e^{-\frac{z^2}{2}} dz  = 0 \\
\int_{C_3} e^{-\frac{z^2}{2}} dz &\to \int_{-\infty}^\infty e^{-\frac{(x-i\omega)^2}{2}} dx = I
\end{align*}

So $\mathcal{F}\left[e^{-\frac{x^2}{2}}\right] = e^{-\frac{\omega^2}{2}}$, meaning the Fourier transform of a Gaussian is a Gaussian in the transformed variable.
\end{example}

\subsection{Convolutions}

Consider two functions $f(x)$ and $g(x)$ with respective Fourier trasnforms $\hat{f}(\omega)$ and $\hat{g}(\omega)$. We define the convolution of $f$ and $g$ as 

\begin{equation}\label{eq:f_convolution}
f(x) \star g(x) = \frac{1}{\sqrt{2\pi}}\intfty g(y)f(x-y)dy
\end{equation}

We can express $f(x-y$ using the definition of the inverse Fourier transform:

\begin{align*}
f(x) \star g(x) &= \frac{1}{\sqrt{2\pi}} \intfty g(y) \left(\intFinv{\hat{f}{\omega}}{(x-y)}{\omega}\right) \\
&= \intFinv{\hat{f}(\omega) \left(\intF{g(y)}{y}{\omega}\right)}{x}{\omega} \\
&= \intFinv{\hat{f}(\omega)\hat{g}(\omega)}{x}{\omega} \\
&= \Finv{\hat{f}(\omega)\hat{g}(\omega)}{x} \\
\intertext{Applying the Fourier transform to both sides:} \\
\F{f(x) \star g(x)}{\omega} &= \hat{f}(\omega)\hat{g}(\omega)
\end{align*}

\begin{example}{}
Consider the differential equation \[f''(x) - f(x) = e^{-x^2}\] The homogeneous solution is \[f_0(x) = Ae^x + Be^{-x}\]

The third particular solution can be found by applying the Fourier transform to each side:
\begin{align*}
\F{f''(x) - f(x)}{\omega} &= \F{e^{-x^2}}{\omega} \\
(-i\omega)^2\hat{f}(\omega) - \hat{f}(\omega) &= \F{e^{-x^2}}{\omega} \\
\hat{f}(\omega) &= -\frac{1}{1+\omega^2}\cdot\F{e^{-x^2}}{\omega} \\
\intertext{So by application of the convolution theorem:} \\
f(x) &= -e^{-x^2} \star \Finv{\frac{1}{1+\omega^2}}{x}
\end{align*}

We can evaluate the inverse Fourier transform using complex analysis, by considering a contour:

\begin{equation*}
C_R = \begin{cases}[0,R]\cup\left\{\frac{R}{2}\left(1 + e^{i\theta}\right)\right\},\theta\in\left[0,-\pi\right) & \text{ for } x \geq 0 \\ [-R,0]\cup\left\{\frac{R}{2}\left(-1 + e^{i\theta}\right)\right\},\theta\in\left[0,\pi\right) & \text{ for } x <\geq> 0 \end{cases}
\end{equation*}

This leads to

\begin{align*}
\Finv{\frac{1}{1+\omega^2}}{x} &= \intFinv{\frac{1}{1+\omega^2}}{x}{\omega} \\
&= \frac{1}{\sqrt{2\pi}} 2\pi i\left(\left.-Res\left[\frac{e^{-i\omega x}}{1+\omega^2}\right]_{x=-i}\right|_{x\geq 0} + \left.Res\left[\frac{e^{-i\omega x}}{1+\omega^2}\right]_{x=i}\right|_{x < 0} \right) \\
&= \sqrt{\frac{\pi}{2}}\left(\left.e^{-x}\right|_{x\geq 0} + \left. e^x\right|_{x<0}\right) \\
&= \sqrt{\frac{\pi}{2}}e^{|x|}
\end{align*}

So 
\begin{align*}
f(x) &= -e^{-x^2} \star \sqrt{\frac{\pi}{2}}e^{-|x|} \\
&= -\frac{1}{2}\intfty e^{-(x-y)^2}e^{-|y|} dy
\end{align*}
\end{example}

\subsection{Parseval's Theorem}

\begin{theorem}{Parseval's Theorem}
$\intfty f^\star(x)g(x)dx = \intfty \hat{f}(\omega)\hat{g}(\omega)d\omega$
\end{theorem}

We see, using a property of the Dirac $\delta$-function \ref{sec:dirac_delta}:

\begin{align*}
\intfty f^\star(x)g(x)dx &= \intfty \left(\Finv{\hat{f}(\omega)}{x}\right)^\star \Finv{\hat{g}(\omega)}{x} dx \\
&= \frac{1}{2\pi}\intfty dx \intfty \hat{f}^\star(\omega)e^{i\omega x} d\omega \intfty \hat{g}(\omega') e^{-i\omega' x} d\omega' \\
&= \intfty \intfty \intfty dx d\omega d\omega' \hat{f}^\star (\omega) \hat{g}(\omega') e^{-ix(\omega' - \omega)}
&= \intfty \intfty d\omega d\omega' \hat{f}^\star(\omega) \hat{g}(\omega') \delta(\omega' - \omega) \\
&= \intfty \hat{f}^\star(\omega)g(\omega) d\omega
\end{align*}

Note that if we take $g(x) = f(x)$, the Parseval theorem shows that normalisation is consered by the Fourier transform.

The Fourier transform is thus a unitary operation in the Hilbert space $\mathbb{L}^2$ of square-integrable functions, and the PArseval theorem is a manifestation of this property.

\begin{example}{$I = \int_{0}^\infty \frac{d\omega}{(a^2 + \omega^2)^2}$}

Define $k = \frac{\omega}{a}$. Thus:

\[ I \to \frac{1}{2a^3}\intfty \frac{dk}{(1 + k^2)^2} \]

We know $\Finv{\frac{1}{1+k^2}}{x} = \sqrt{\frac{\pi}{2}} e^{|x|}$, hence by Parseval's theorem:

\begin{align*}
\intfty \frac{dk}{(1+k^2)^2} &= \intfty \left(\sqrt{\frac{\pi}{2}}e^{-|x|}\right)^2 dx \\
&= \pi\int_{0}^\infty e^{-2x} dx \\
&= \frac{\pi}{2}
\end{align*}

So $I = \frac{\pi}{4a^3}$
\end{example}



\subsection{Solving Integral Equations}

The Fourier transform can also be used to solve integral equations that involve convolution of functions. Consider

\[ h(x) = e^{i3x} + \intfty e^{-|y|}h(x-y)dy \]

Noticing that the integral on the right-hand side is a convolution, we can apply the Fourier transform to obtain

\begin{align*}
\hat{h}(\omega) &= \sqrt{2\pi} \delta(\omega + 3) + \frac{2}{1+w^2} \hat{h}(\omega) \\
\intertext{ which can be written }
&= \sqrt{2\pi} \delta(\omega + 3)\left(1 + \frac{2}{1+w^2} \hat{h}(\omega)\right)^{-1}
\end{align*}

Inverting this relation gives
\begin{align*}
h(x) &= \intF{\sqrt{2\pi} \delta(\omega + 3)\left(1 + \frac{2}{1+w^2} \hat{h}(\omega)\right)^{-1}}{x}{\omega} \\
&= \frac{5}{4}e^{i3x}
\end{align*}

\subsection{Discrete Fourier Transforms} \index{discrete Fourier transform}

For a function $h(t)$ with period $T$, we can perform a set of measurements $h_j$ within a period. We will consider $h_j$ to be an even number of points $2N$ sampled at times $t_j = j\frac{T}{2N}$.

The discrete Fourier transform (DFT) for this set of data is

\begin{equation}\label{eq:dft}
\hat{h}(\omega_p) = \frac{1}{\sqrt{2N}}\sum_{j=0}^{2N-1} h_j e^{i\omega_p t_j}
\end{equation}

where $\omega_p = \frac{2\pi p}{T}$ with $0 \leq p \leq 2N-1$.

Notice that this decomposition includes as many frequencies as time measurements were performed. We can invert this relation:

\begin{equation*}\label{eq:inv_dft}
h^{DFT}(t) = \frac{1}{\sqrt{2N}}\sum_{p=0}^{2N-1} \hat{h}_p e^{-i\omega_p t}
\end{equation*}

Note that $h^{DFT}$ is defined for all $t$, with $h^{DFT} \to h$ as $N\to\infty$.

\subsection{The Fourier Matrix}\index{Fourier matrix}

We can express the above procedure in terms of the Fourier matrix:

\begin{equation*}
\begin{pmatrix} \hat{h}_1 \\ \vdots \\ \hat{h}_{2N-1} \end{pmatrix} = F(2N)_{p_j} \begin{pmatrix} h_1 \\ \vdots \\ h_{2N-1} \end{pmatrix}
\end{equation*}

where
\begin{equation}
F(2N)_{p_j} = \left(\frac{e^{i\omega_p t_j}}{\sqrt{2N}}\right) = \left(\frac{e^{i\pi p j/N}}{\sqrt{2N}}\right)
\end{equation}
is the Fourier matrix which provides the basis of plane waves in which the signal is expanded.

\begin{example}{}
Consider the periodic signal $h(t) = \cos{t}$, with period $T = 2\pi$. Using the following measurements, N = 2:


\begin{tabular}{c || c | c | c}
$t$ & $h(t)$ & $p$ & $\omega_p$ \\
\hline
$0$ & $1$ & $0$ & $\frac{0}{T} = 0$ \\
$\frac{\pi}{2}$ & $0$ & $1$ & $\frac{2\pi}{T} = 1$ \\
$\pi$ & $-1$ & $2$ & $\frac{4\pi}{T} = 2$ \\
$\frac{3\pi}{2}$ & $0$ & $3$ & $\frac{6\pi}{T} = 3$
\end{tabular}

So

\begin{equation*}
\begin{pmatrix} \hat{h}_0 \\ \hat{h}_1 \\ \hat{h}_2 \\ \hat{h}_3 \end{pmatrix} = \frac{1}{2}\begin{pmatrix} 1 & 1 & 1 & 1 \\ 1 & i & -1 & -i \\ 1 & -1 & 1 & -1 \\ 1 & -i & -1 & i\end{pmatrix}\begin{pmatrix} 1 \\ 0 \\ -1 \\ 0\end{pmatrix}
\end{equation*}

We can now apply the inverse DFT to obtain

\begin{align*}
\hat{h}(t) &= \frac{1}{2}\sum_{p=0}^{2N-1} \hat{h}_p e^{-i\omega_p t} \\
&= \frac{1}{2}\left( 0\cdot e^{-0\cdot it} + 1\cdot e^{-1\cdot it} + 0\cdot e^{-2\cdot it} + 1\cdot e^{-3\cdot it}\right) \\
\intertext{Given that the function is real:} \\
h(t) &= \frac{1}{2}\left(\cos{t} + \cos{3t}\right)
\end{align*}
\end{example}

\section{Laplace Transforms} \index{Laplace transform}

The Laplace transform is an integral transform with kernel $k(t,s) = e^{-ts}$, defined:

\begin{equation}
\L{f(t)}{s} \equiv \bar{f}(s) = \intL{f(t)}{t}{s}
\end{equation}

which is typically defined for $s>0$.

We have the following standard results:

\begin{enumerate}
	\item{}
\end{enumerate}

\begin{example}
\begin{align*}
\L{1}{s} &= \intL{}{t}{s} \\
&= \frac{1}{s} \, s>0
\end{align*}
\end{example}

\begin{example}
\begin{align*}
\L{e^{at}}{s} &= \intL{e^{at}}{t}{s} \\
&= \zintfty e^{-(s-a)t dt \\
&= \frac{1}{s-a} \, s>a
\end{align*}
\end{example}

\subsection{Link to Fourier Transform}

We can relate the Laplace transform to the Fourier transform by assuming $s=x+iy$:

\begin{align}
\L{f(t)}{s} &= \intL{f(t)}{t}{s} \nonumber \\
&= \zintfty f(t) e^{-(x+iy)t} dt \nonumber \\
&= \zintfty f(t) e^{-xt} e^{-iyt} dt \nonumber \\
&= \intfty f(t) e^{-xt} H(t)e^{iyt} dt \nonumber \\
&= \Finv{f(t)H(t)e^{-xt}}{y}
\end{align}

\begin{example}
\begin{align*}
\L{\frac{df}{dt}}{s} &= \intL{\frac{df}{dt}}{t}{s} \\
&= \underbrace{\left. f(t)e^{-st}\right|^\infty_0}{-f(0) \text{ by convergence.}} - \zintfty f(t)\frac{\partial e^{-st}}{\partial t} dt \\
&= -f(0) + s\zintfty f(t) e^{-st} dt \\
&= s\L{f(t)}{s} - f(0)
\end{align*}
\end{example}

In general:
\begin{equation*}
\L{\frac{d^n f}{dt^n}}{s} = s^n\L{f(t)}{s} - \sum_{k=0}^{n-1} s^{n-1-k}f^{(k)}(0)
\end{equation*}

\begin{example}{$f(t) = \sinh{kt} = \frac{1}{k}\frac{\partial \cosh{kt}}{\partial t}$}
\begin{align*}
\L{\sinh{kt}}{s} &= \frac{1}{k}\L{\frac{\partial}{\partial t}\cosh{kt}}{s} \\
&= \frac{s}{k}\L{\cosh{kt}}{s} - \frac{1}{k}\cosh{0} \\
&= \frac{1}{k}\left(\frac{s^2}{s^2 - k^2} - 1\right) \\
&= \frac{k}{s^2 - k^2} \, s>|k|
\end{align*}
\end{example}

The Laplace transform can also be a useful tool in solving integrals.

\begin{example}{$f(t) = \zintfty \frac{\sin{tx}{x}}dx$}
\begin{align*}
\bar{f}(s) &= \L{\zintfty \frac{\sin{tx}{x}}}{s} \\
&= \zintfty dt \zintfty\frac{\sin{tx}}{x}e^{-st} dx \\
&= \zintfty \L{\sin{tx}}\frac{dx}{x} \\
&= \zintfty \frac{1}{s^2 + x^2} dx \\
&= \frac{\pi}{2s}
\end{align*}

So for $t>0$ we have $f(t) = \frac{\pi}{2}$. For $t<0$ we use $\sin{-t} = -\sin{t}$ to obtain $f(t) = -\frac{\pi}{2}$. Hence $f(t) = \frac{\pi}{2}\sgn{t}
\end{example}


\subsection{Laplace Transforms of Periodic Functions}


\section{The Dirac $\delta$-function}\label{sec:dirac_delta}

The Dirac $\delta$-function is commonly defined as follows:

\begin{enumerate}
	\item $\delta(x) = 0$ if $x\neq 0$
	\item $\int_{-\infty}^\infty \delta(x) dx = 1$
	\item $\int_{-\infty}^\infty \delta(x) f(x) dx = f(0)$
\end{enumerate}

In order to obtain a more rigorous definition, we consider a sequence of functions:

\begin{equation*}
f_n = \frac{n}{\sqrt{\pi}}e^{-n^2x^2}
\end{equation*}

As $n\to\infty$, $f_n$ replicates the behaviour of the $\delta$-function.

Similarly, we can define it as the limit as $n\to\infty$ of

\begin{equation*}
g_n(x) = \frac{n}{\pi}\frac{1}{1+n^2x^2}
\end{equation*}

We see:

\begin{enumerate}
	\item{\[\lim_{n\to\infty}g_n(x) = \begin{cases} 0 & \text{ if } x\neq0 \\ \infty & \text{ if } x=0\end{cases}\]}
	\item{
	\begin{align*} \lim_{n\to\infty}\int_{-\infty}^\infty g_n(x)dx &= \lim_{n\to\infty}\int_{-\infty}^\infty \frac{n}{\pi}\frac{1}{1+n^2x^2}dx \\
	&= \frac{2\cancel{\pi} i}{\cancel{\pi}}Res\left(\frac{1}{n^2(z+\frac{i}{n})(z-\frac{i}{n})}\right)_{z=\frac{i}{n}} \\
	&= 1
	\end{align*}}
	\item{
	\begin{align*} \lim_{n\to\infty}\int_{-\infty}^\infty g_n(x)f(x)dx &= \lim_{n\to\infty}\int_{-\infty}^\infty \frac{n}{\pi}\frac{f(x)}{1+n^2x^2}dx \\
	&= \frac{2\cancel{\pi} i}{\cancel{\pi}}Res\left(\frac{f(x)}{n^2(z+\frac{i}{n})(z-\frac{i}{n})}\right)_{z=\frac{i}{n}} \\
	&= f\left(\frac{i}{n}\right) \stackrel{n\to\infty}{\longrightarrow} f(0)
	\end{align*}}
\end{enumerate}

There are many other similar sequences of functions which we could use. The fact that we must define the Dirac $\delta$-function in this way means that it is a distribution.

\subsection{Properties of the Dirac $\delta$-function}
\begin{enumerate}
	\item{$\delta(-x) = \delta(x)$}
	\item{$\delta(x-a)f(x) = \delta(x-a)f(a)$ if $f$ is continuous at $x=a$}
	\item{If a function $g(x)$ has at most simple zeros $x_i$ then \begin{equation*}\delta(g(x)) = \sum_{x_i} \frac{\delta(x-x_i)}{\left|g'(x_i)\right|}\end{equation*}}
	\item{$\delta(ax) = \frac{\delta(x)}{\|a\|}$}
	\item{The Dirac $\delta$-function is differentiable, with: \begin{align*}\int_{-\infty}^\infty \delta '(x)f(x)dx &= \left. f(x)\delta(x)\right|_{-\infty}^\infty - \int_{-\infty}^\infty \delta(x) f'(x) dx \\ &= -f'(0) \\ \intertext{where $f(x)$ vanishes at $\pm\infty$. Similarly, for higher orders:} \intfty \delta^n (x)f(x)dx = (-1)^n f^{(n)}(0) \end{align*}}
	\item{$H'(x) = \delta(x)$, where $H(x)$ is the Heaviside step function, defined: \begin{equation} H(x) = \begin{cases} 1 & x\geq 0 \\ 0 & x < 0 \end{cases}\end{equation}}
\end{enumerate}
\subsection{Fourier Representation of the Dirac $\delta$-function}

We see:

\begin{align*}
\F{\delta(x)}{\omega} &= \intF{x}{\omega}{\delta(x)} \\
&= \frac{1}{\sqrt{2\pi}}
\end{align*}

Applying the inverse transform, we obtain:
\begin{align*}
\Finv{\hat{\delta}(\omega)}{x} &= \intFinv{x}{\omega}{\frac{1}{\sqrt{2\pi}}} \\
&= \frac{1}{2\pi}\intfty{e^{-i\omega x} d\omega}
\end{align*}

So we can show:
\begin{enumerate}
	\item{$\F{1}{\omega} = \sqrt{2\pi}\delta(\omega)$}
	\item{$\F{e^{i\alpha x}}{\omega} = \intF{e^{i\alpha x}}{x}{\omega} = \sqrt{2\pi}\delta(\omega + \alpha)$}
\end{enumerate}


\newpage
\printindex
\end{document}
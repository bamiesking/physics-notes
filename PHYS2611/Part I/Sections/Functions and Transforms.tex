\subsection{Fourier Series} %% Above section (?)




\subsection{The Dirac Distribution}
The Dirac $\delta$-function is not a function. It is considered a generalised function or distribution. There are lots of ways to think about what it is mathematically and in the context of what we'll be using it for, integrals.
It is the limit of a rectangle shaped graph (or gaussian..) where the height tends to infinity and the width tends to zero centered at the origin. However, it has the property that
%%%Figure of the distribution! 
\begin{equation}
\int_{-\infty}^{+\infty} \! \delta(x) \textrm{d}x = 1 
\end{equation}
This is its most defining property and can be seen clearly in all of its properties.
\begin{itemize}
\item \begin{math}\delta(x - a) = 0 \forall x \neq 0\end{math} As defined above, our function is zero everywhere execpt at the origin.
\item \begin{math} \int_{-\infty}^{+\infty} \! \delta(x-a)f(x) \textrm{d}x = f(a) \end{math} Hopefully makes sense. The resulting function from the product is simply the value of the function where $\delta(x)$ is one. Note that $a$ would have to be in the integral limits. 
\end{itemize}
One way of defining it is by a sequence given by
\begin{equation}
\delta_{n}(t-x) = \frac{\sin(n[t-x])}{\pi(t-x)} = \int_{-n}^{n} \! \frac{exp(-i\omega(t-x))}{2\pi} \textrm{d}\omega 
\end{equation}
This is slightly convoluted but is a useful formulation when using transforms which we will find out later on. By taking the limit of above as $n \to \infty$, we arrive at 
\begin{equation}
\delta(t-x) = \frac{1}{2\pi} \int_{-\infty}^{+\infty} \! exp(-i\omega(t-x)) \textrm{d}\omega
\end{equation}
Some further properties to note 
\begin{itemize}
\item \begin{math}\delta(x) = \delta(-x) \end{math} It is even!
\item  \begin{math} \delta(g(x)) =\sum_{a_{i}} \frac{\delta(x-a_{i})}{\abs{g'(a_{i})}}\end{math} where $a_{i}$ is each (distinct) root of the function g(x) (and $g'(a_{i})\neq0$).
\end{itemize}
A close relative of this is the Heaviside step function. It is 0 for all negative numbers and 1 for all postive numbers. It is usually defined as 0.5 at 0 ('half maximum convention') and/or have a discontinuity at 0. Once can notice that the Heaviside step function is the cumulative distribution function of the Dirac $\delta$-function. It is then written that
\begin{equation}
H(x) = \int_{-\infty}^{+\infty} \! \delta(x)f(x) \textrm{d}x \implies H'(x) = \delta(x)
\end{equation}\par 
\title{\large{Exercises}}
\begin{enumerate}
\item What is \begin{math} \int_{-9}^{0} \! \delta(x-1) \textrm{d}x \end{math} ?
\item What about \begin{math} \int_{-9}^{2} \! \delta(x-1)  \textrm{d}x \end{math} ?
\item And finally  \begin{math} \int_{-9}^{2} \! \delta(x-1)\delta(x+3) \textrm{d}x \end{math} ?
\end{enumerate}
\begin{figure}
\centering
\includegraphics[scale=0.2]{"Figures/Hside"}
\captionsetup{justification=centering}
\caption{Recommeded break reading: \url{https://physicstoday.scitation.org/doi/10.1063/PT.3.1788} `\textit{Mathematics is an experimental science and definitions do not come first, but later on. [..] Shall I refuse my dinner because I do not fully understand the process of digestion?}' - O. Heaviside. The department's approach to the course. } %%%READ: image may not appear on built in pdf viewer. When printed, images appear%%%
\label{fig:Hside}
\end{figure}
\newpage
\subsection{The Fourier and Laplace Transform}
An integral transfrom is defined by the equation: 
\begin{equation}
\int_{-\infty}^{+\infty} \! k(x, y) f(x) \textrm{d}x = I[f(x)](y)
\end{equation}
Let's have a look at what is going on. We have an 'input' function f(x) that is put in this transform and what we get is a function dependent on the variable y. The k(x, y) is said to be the \textit{kernel} of the transform analgous to the kernel in linear algebra. It is a linear operator. This is true since integration is a linear operator. [Quick reminder: \begin{math} I[af(x) + bg(x)] = aI[f(x)] + bI[g(x)]\end{math}]
If I is given, we can introduce the inverse operator of $I$, $I^{-1}$ [\begin{math} I[f]=g and I^{-1}[g] = f\end{math}] 
The two looked at in this course are the \textit{Laplace Transform} and the \textit{Fourier Transform}
Here is the Fourier Transform
\begin{equation}
\mathcal{F}[f(t)](\omega) = \hat{\mathcal{f}(w)} = \frac{1}{\sqrt{2 \pi}} \int_{-\infty}^{+\infty} \! f(t) exp(-i\omega t) \textrm{d}t
\end{equation}
Before explaining (if ever) where this comes from ~\ref{fig:Hside}, let's take a look at a few of its properties.
The integral exists if
\begin{itemize}
\item f has a finite number of discontinuities
\item \begin{math} \int_{-\infty}^{+\infty} \! \abs{f(t)} \textrm{d}t is finite/\end{math}
\item If f is continous we can define the inverse Fourier transform as follows;
\begin{math} \mathcal{F}^{-1}[\hat{f(\omega)}](t) = f(t) = \frac{1}{\sqrt{2\pi}} \int_{-\infty}^{+\infty}\hat{f}(\omega)exp(i\omega t) \textrm{d}\omega
\end{math}
\end{itemize}
As a brief aside, the convention for the fourier transform can vary with some conventions have $\omega$ as $2\pi \mu v$ or having no 2$\pi$.\par
So why do we use this transform? A lot of functions are not periodic and whilst it would be easy to use the fourier series to decompose a function sometimes we cannot do that. We can imagine that the function defined has an infinite period or interval. Do you remember the complex (general) definition of the Fourier series?
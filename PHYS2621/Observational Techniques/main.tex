\documentclass{physics_notes}

\title{Stars and Galaxies: Observartional Techniques}
\author{St Aidan's Physics Society}
\date{\today}

\begin{document}

\maketitle


\tableofcontents
\newpage

\section{Preface}
This is L2 Stars and Galaxies. If you've done introduction to astronomy in L1, it will certainly help your understanding of many concepts discussed in the lectures. However, you will fine if you haven't done this course. \par Physics as problem solving is used a lot in this course. Many derived equations and models for processes begin with a few assumptions and approximations. Perhaps because we cannot recreate some environments in the lab.... However, it is still reassuring to know that physics \textit{works} even when simple models are created. In the context of astronomy, when we make observations and measurements of the night sky, we might have to adjust our model based on what we see and even completley rework from the ground up.



\section{Introduction: Observation}
\subsection{Basic Principles}
\title{The Night Sky}
To start, we'll look at what one first sees when looking at the night sky: Stars. We first need some way of measuring how bright a star is. The problem is they are not the same distance away. We cannot have the notion of the 'absolute brightness' of a star without considering how much that light has spread out. Before we look at this, let's first consider what the star has. Let's define some Luminosity, $L$, as a measure of its energy per unit time. Note that this energy is the sum over all wavelengths of the EM spectrum. This way, the value is independent of it's distance; its intrinsic to the star. To give some sense of scale, the Sun has $L=3.9$ $\times$ $10^{26}$ ms${}^{-1}$. We know that the intensity of a light source falls off as $\frac{1}{r^{2}}$ and due to spherical geometry we can write:
\begin{equation}
I = \frac{L}{4\pi r^{2}}
\end{equation}
To further add to this, our instruments cannot measure over the whole range of wavelengths. We have specific instruments for particular 'bands' of the spectrum. This is discussed in $3.1 The EM Spectrum$. A particular instrument may measure the 'V-band' of the spectrum. This means the intensity is measured \textit{at that wavelength}. For the 'V-band', this corresponds to 550 nm with a width of 90 nm. This is denoted
\begin{equation}
I_{V} = \frac{L_{V}}{4 \pi r^{2}}
\end{equation}
\par
Astronomers use magnitudes to express intensities. Historically, because of the greeks but intuitively because we can imagine that stars will have a big range of intensities and may span quite a few powers of ten. Mathematically, we \textit{define} the (relative) apparent magnitude, $m$, to be
\begin{equation}
m_{a} - m{b} = -2.5log(\frac{I_{a}}{I_{b}})
\end{equation}
This is important! It is \textit{defined} this way such that if a source, $m_{a}$, has $I_{a} = 100I_{b}$ then it follows that $m_{a} - m_{b} = 5$. Make sure this makes sense, try it yourself. \par
Bringing this definition back to before, we need a sense of the 'absolute magnitude' of a star. This is so we can compare them at equal footing. By convention we place the star at $D=10$ pc from the Sun. By using the flux, $f$, of a star at a distance d, we can give a measure of the Luminosity \textit{at} a distance D. 
\begin{equation}
\frac{L}{f} = (\frac{d}{10})^{2}
\end{equation}
The flux \textit{is} the luminosity at a given distance so the ratio of these gives a ratio of the distances sqaured. The difference between the absolute magnitude and apparent magnitude is called the distance modulus. 
\begin{equation}
m - M = 5log(d) -5
\end{equation}
\title{\large{Questions}}
\begin{enumerate}
	\item Equation (1.1) still works for intensities in a band i.e. $I_{a}$. Consider the apparent magnitudes of the \textit{same} star using different bands. We have $m_{B} - m{V} = -2.5log(\frac{I_{B}}{I_{V}})$. What does it mean when this value is close to $0^{-}$? What about close to $0^{+}$ or greater? [The V band is at 550 nm with a width of 90 nm and the B band is at 440 with a width of 100 nm] \par (Hint: This value is called the \textit{color} of a source!)
	\item Derive from (5) using (4) and (3).\par (Hint: Which of L, f corresponds to $I_a$ etc. in (3)?) 
	\item What does it tell you about a star if it has a distance modulus of 0? What about -5? or +5? 
\end{enumerate}
\title{\large{Other Aspects}}
We'll need a common measure of distance as well. You may have seen it before undergraduate physics or at L1. The Parsec is \textit{defined} as the distance at which one astronomical unit subtends an angle of one arcsecond. We'll break this down briefly to see what it corresponds to. You'll recognise that the average distance between the Earth and the Sun is 1 AU or ~1.5 $\times$ 10${}^{11}$ m. As the Earth orbits the Sun, there will be apparent parallax motion of the near star you are observing.
\begin{figure}
\centering
\begin{tikzpicture}
\draw[dashed] (0,0) circle [radius=2];
\draw[dashed] (0,0) -- (10,0); 
\draw (0,-2) -- (10,0);
\draw (0,2) -- (10,0);
\draw[fill, blue] (0,2) circle node[above] {Earth} [radius=0.5];
\draw[fill, yellow] (10,0) circle node[right] {Star} [radius=0.5];
\draw[fill, orange] (0,0) circle node[below] {Sun} [radius=0.5];
\end{tikzpicture}
\caption{Figure showing the derivation of 1 parsec}
\end{figure}
We'll also need some coordinate system for describing stars on the night sky.











\subsection{Telescopes}
\subsection{The Atmosphere}



\section{The Measurements}
\subsection{Measuring Stars}
\subsection{Detectors and photometry}
\subsection{Spectroscopy}



\section{Further Techniques}
\subsection{The EM Spectrum}
\subsection{Instrumentation}

\end{document}